\documentclass{ctexart}
\usepackage{paracol}
\usepackage{newtxtext, geometry, amsmath, amssymb,yhmath}
\usepackage{graphicx}
\usepackage{amsthm} % 设置定理环境,要在amsmath后用
\usepackage{xpatch} % 用于去除amsthm定义的编号后面的点
\usepackage[strict]{changepage} % 提供一个 adjustwidth 环境
\usepackage{multicol} % 用于分栏
\usepackage{tikz}
\usetikzlibrary{arrows.meta,decorations.markings}
\usepackage[fontsize=14pt]{fontsize} % 字号设置
\usepackage{esvect} % 实现向量箭头输入(显示效果好于\vec{}和\overrightarrow{}),格式为\vv{⟨向量符号⟩}或\vv*{⟨向量符号⟩}{⟨下标⟩} 
\usepackage[
            colorlinks, % 超链接以颜色来标识,而并非使用默认的方框来标识
            linkcolor=mlv,
            anchorcolor=mlv,
            citecolor=mlv % 设置各种超链接的颜色
            ]
            {hyperref} % 实现引用超链接功能
\usepackage{tcolorbox} % 盒子效果
\tcbuselibrary{most} % tcolorbox宏包的设置,详见宏包说明文档
\tcbuselibrary{breakable=forced}
\geometry{a4paper,centering,scale=0.8}

% ————————————————————————————————————自定义符号————————————————————————————————————
\newcommand{\。}{.} % 示例:将指令\。定义为全角句点。
\newcommand{\zte}{\mathrm{e}} % 正体e
\newcommand{\ztd}{\mathrm{d}} % 正体d
\newcommand{\zti}{\mathrm{i}} % 正体i
\newcommand{\ztj}{\mathrm{j}} % 正体j
\newcommand{\ztk}{\mathrm{k}} % 正体k
\newcommand{\CC}{\mathbb{C}} % 黑板粗体C
\newcommand{\QQ}{\mathbb{Q}} % 黑板粗体Q
\newcommand{\RR}{\mathbb{R}} % 黑板粗体R
\newcommand{\ZZ}{\mathbb{Z}} % 黑板粗体Z
\newcommand{\NN}{\mathbb{N}} % 黑板粗体N
% ————————————————————————————————————自定义符号————————————————————————————————————

% ————————————————————————————————————自定义颜色————————————————————————————————————
\definecolor{mlv}{RGB}{40, 137, 124} % 墨绿
\definecolor{qlv}{RGB}{240, 251, 248} % 浅绿
\definecolor{slv}{RGB}{33, 96, 92} % 深绿
\definecolor{qlan}{RGB}{239, 249, 251} % 浅蓝
\definecolor{slan}{RGB}{3, 92, 127} % 深蓝
\definecolor{hlan}{RGB}{82, 137, 168} % 灰蓝
\definecolor{qhuang}{RGB}{246, 250, 235} % 浅黄
\definecolor{shuang}{RGB}{77, 82, 59} % 深黄
\definecolor{qzi}{RGB}{240, 243, 252} % 浅紫
\definecolor{szi}{RGB}{49, 55, 70} % 深紫
% 将RGB换为rgb,颜色数值取值范围改为0到1
% ————————————————————————————————————自定义颜色————————————————————————————————————

% ————————————————————————————————————盒子设置————————————————————————————————————
% tolorbox提供了tcolorbox环境,其格式如下:
% 第一种格式:\begin{tcolorbox}[colback=⟨背景色⟩, colframe=⟨框线色⟩, arc=⟨转角弧度半径⟩, boxrule=⟨框线粗⟩]   \end{tcolorbox}
% 其中设置arc=0mm可得到直角;boxrule可换为toprule/bottomrule/leftrule/rightrule可分别设置对应边宽度,但是设置为0mm时仍有细边,若要绘制单边框线推荐使用第二种格式
% 方括号内加上title=⟨标题⟩, titlerule=⟨标题背景线粗⟩, colbacktitle=⟨标题背景线色⟩可为盒子加上标题及其背景线
\newenvironment{kuang1}{
    \begin{tcolorbox}[breakable,colback=hlan, colframe=hlan, arc = 1mm]
    }
    {\end{tcolorbox}}
\newenvironment{kuang2}{
    \begin{tcolorbox}[breakable,colback=hlan!5!white, colframe=hlan, arc = 1mm]
    }
    {\end{tcolorbox}}
% 第二种格式:\begin{tcolorbox}[enhanced, colback=⟨背景色⟩, boxrule=0pt, frame hidden, borderline={⟨框线粗⟩}{⟨偏移量⟩}{⟨框线色⟩}]   {\end{tcolorbox}}
% 将borderline换为borderline east/borderline west/borderline north/borderline south可分别为四边添加框线,同一边可以添加多条
% 偏移量为正值时,框线向盒子内部移动相应距离,负值反之
\newenvironment{kuang3}{
    \begin{tcolorbox}[breakable,enhanced, colback=hlan!5!white, boxrule=0pt, frame hidden,
        borderline south={0.5mm}{0.1mm}{hlan}]
    }
    {\end{tcolorbox}}
\newenvironment{lvse}{
    \begin{tcolorbox}[breakable,enhanced, colback=qlv, boxrule=0pt, frame hidden,
        borderline west={0.7mm}{0.1mm}{slv}]
    }
    {\end{tcolorbox}}
\newenvironment{lanse}{
    \begin{tcolorbox}[breakable,enhanced, colback=qlan, boxrule=0pt, frame hidden,
        borderline west={0.7mm}{0.1mm}{slan}]
    }
    {\end{tcolorbox}}
\newenvironment{huangse}{
    \begin{tcolorbox}[breakable,enhanced, colback=qhuang, boxrule=0pt, frame hidden,
        borderline west={0.7mm}{0.1mm}{shuang}]
    }
    {\end{tcolorbox}}
\newenvironment{zise}{
    \begin{tcolorbox}[breakable,enhanced, colback=qzi, boxrule=0pt, frame hidden,
        borderline west={0.7mm}{0.1mm}{szi}]
    }
    {\end{tcolorbox}}
% tcolorbox宏包还提供了\tcbox指令,用于生成行内盒子,可制作高光效果
\newcommand{\hl}[1]{
    \tcbox[on line, arc=0pt, colback=hlan!5!white, colframe=hlan!5!white, boxsep=1pt, left=1pt, right=1pt, top=1.5pt, bottom=1.5pt, boxrule=0pt]
{\bfseries \color{hlan}#1}}
% 其中on line将盒子放置在本行(缺失会跳到下一行),boxsep用于控制文本内容和边框的距离,left、right、top、bottom则分别在boxsep的参数的基础上分别控制四边距离
% ————————————————————————————————————盒子设置————————————————————————————————————

% ————————————————————————————————————自定义字体设置————————————————————————————————————
\setCJKfamilyfont{xbsong}{方正小标宋简体}
\newcommand{\xbsong}{\CJKfamily{xbsong}}
% CTeX宏集还预定义了\songti、\heiti、\fangsong、\kaishu、\lishu、\youyuan、\yahei、\pingfang等字体命令
% 由于未知原因,一些设备可能无法调用这些字体,故此文档暂时未使用
% ————————————————————————————————————自定义字体设置————————————————————————————————————

% ————————————————————————————————————各级标题设置————————————————————————————————————
\ctexset{
    % 修改 section。
    section={   
    % 设置标题编号前后的词语,使用name={⟨前部分⟩,⟨后部分⟩}参数进行设置。    
        name={\textbf{第},\textbf{章}\hspace{18pt}},
    % 使用number参数设置标题编号,\arabic设置为阿拉伯数字,\chinese设置为中文,\roman设置为小写罗马字母,\Roman设置为大写罗马字母,\alph设置为小写英文,\Alph设置为大写英文。
        number={\textbf{\chinese{section}}},
    % 参数format设置标题整体的样式。包括标题主题、编号以及编号前后的词语。
    % 参数format还可以设置标题的对齐方式。居中对齐\centering,左对齐\raggedright,右对齐\hfill
        format=\color{hlan}\centering\zihao{2}, % 设置 section 标题为正蓝色、黑体、居中对齐、小二号字
    % 取消编号后的空白。编号后有一段空白,参数aftername=\hspace{0pt}可以用来控制编号与标题之间的距离。
        aftername={}
    },
    % 修改 subsection。
    subsection={   
        name={\S \hspace{6pt}, \hspace{8pt}},
        number={\arabic{section}.\arabic{subsection}},
        format=\color{hlan}\centering\zihao{-2}, % 设置 subsection 标题为黑体、三号字
        aftername=\hspace{0pt}
    },
    subsubsection={   
        name={,、},
        number={\chinese{subsubsection}},
        format=\raggedright\color{hlan}\zihao{3},
        aftername=\hspace{0pt}
    },
    part={   
        name={第,部分},
        number={\chinese{part}},
        format=\color{white}\centering\bfseries\zihao{-1},
        aftername=\hspace{1em}
    }
}
% 各个标题设置的各行结尾一定要记得写逗号!!!!!!!!!!!!!!!!!!!!!!!!!!!!!!!!!!!!!!!!!!
% ————————————————————————————————————各级标题设置————————————————————————————————————

% ————————————————————————————————————定理类环境设置————————————————————————————————————
\newtheoremstyle{t}{0pt}{}{}{\parindent}{\bfseries}{}{1em}{} % 定义新定理风格。格式如下:
%\newtheoremstyle{⟨风格名⟩}
%                {⟨上方间距⟩} % 若留空,则使用默认值
%                {⟨下方间距⟩} % 若留空,则使用默认值
%                {⟨主体字体⟩} % 如 \itshape
%                {⟨缩进长度⟩} % 若留空,则无缩进;可以使用 \parindent 进行正常段落缩进
%                {⟨定理头字体⟩} % 如 \bfseries
%                {⟨定理头后的标点符号⟩} % 如点号、冒号
%                {⟨定理头后的间距⟩} % 不可留空,若设置为 { },则表示正常词间间距;若设置为 {\newline},则环境内容开启新行
%                {⟨定理头格式指定⟩} % 一般留空
% 定理风格决定着由 \newtheorem 定义的环境的具体格式,有三种定理风格是预定义的,它们分别是:
% plain: 环境内容使用意大利斜体,环境上下方添加额外间距
% definition: 环境内容使用罗马正体,环境上下方添加额外间距
% remark: 环境内容使用罗马正体,环境上下方不添加额外间距
\theoremstyle{t} % 设置定理风格
\newtheorem{dyhj}{\color{slv} 定义}[subsection] % 定义定义环境,格式为\newtheorem{⟨环境名⟩}{⟨定理头文本⟩}[⟨上级计数器⟩]或\newtheorem{⟨环境名⟩}[⟨共享计数器⟩]{⟨定理头文本⟩},其变体\newtheorem*不带编号
\newtheorem{dlhj}{\color{shuang} 定理}[subsection]
\newtheorem{lthj}{\color{szi} }
\newtheorem*{tmhj}{\color{slan} 解}
\newtheorem*{jiehj}{\color{slan} 解}
\newtheorem*{zmhj}{\color{slan} 证明}
\newtheorem*{smhj}{\color{slan} }
\newenvironment{dy}{\begin{lvse}\begin{dyhj}}{\end{dyhj}\end{lvse}}
\newenvironment{zm}{\begin{lanse}\begin{zmhj}}{$\hfill \square$\end{zmhj}\end{lanse}}
\newenvironment{dl}{\begin{huangse}\begin{dlhj}}{\end{dlhj}\end{huangse}}
\newenvironment{lt}{\begin{zise}\begin{lthj}}{\end{lthj}\end{zise}}
\newenvironment{tm}{\begin{lanse}\begin{tmhj}}{\end{tmhj}\end{lanse}}
\newenvironment{sm}{\begin{lanse}\begin{smhj}}{\end{smhj}\end{lanse}}
% ————————————————————————————————————定理类环境设置————————————————————————————————————

% ————————————————————————————————————目录设置————————————————————————————————————
\setcounter{tocdepth}{4} % 设置在 ToC 的显示的章节深度
\setcounter{secnumdepth}{3} % 设置章节的编号深度
% 数值可选选项:-1 part 0 chapter 1 section 2 subsection 3 subsubsection 4 paragraph 5 subparagraph
\renewcommand{\contentsname}{\bfseries 目\hspace{1em}录}
% ————————————————————————————————————目录设置————————————————————————————————————

\everymath{\displaystyle} % 设置所有数学公式显示为行间公式的样式

\begin{document}

\quad\\
{
\Huge \bfseries 
\begin{kuang3}
    \color{hlan}
    \centering
    复变函数论第八次作业\\
    20234544 毛华豪
\end{kuang3}
}

\thispagestyle{empty} % 取消这一页的页码

% ————————————————————————————————————页码设置————————————————————————————————————
% \pagenumbering{Roman}
% \setcounter{page}{1}

% \begin{multicols}{2} % 设置环境中内容为两栏。数字为栏数
%     {
%     \tableofcontents
%     }
% \end{multicols}

% \newpage
% \setcounter{page}{1}
% \pagenumbering{arabic}
% ————————————————————————————————————页码设置————————————————————————————————————

% 指令\addcontentsline{toc}{⟨章节层级⟩}{⟨标题名⟩}可将“⟨标题名⟩”加入目录。如为无章节层级的标题,可先用\phantomsection指令添加section分级

% \begin{kuang1}
%     \part{第一部分的标题}
% \end{kuang1}

% \begin{kuang2}
%     \section{第一章的标题}
% \end{kuang2}

% \begin{kuang3}
%     \subsection{第一节的标题}
% \end{kuang3}

%\subsubsection{第一个大标题}


% \begin{dy} % 开始定义环境,格式为\begin{⟨环境名⟩}[⟨定理名⟩]
% 测试
% $$\int_{a}^{b} f(x)dx = F(b)-F(a) $$
% \end{dy}

% \begin{dl}
% 测试
% $$\int_{a}^{b} f(x)dx = F(b)-F(a) $$
% \end{dl}

% \begin{lt}
% 测试
% $$\int_{a}^{b} f(x)dx = F(b)-F(a) $$
% \end{lt}
Task1:
\begin{zm}
    这里仅仅需要证明不等式,所以考虑放缩相应的值。设$f(z)=x^2+iy^2$由于:
    \begin{align*}
        \left|\int_{L}^{}f(z)dz\right|\le \int_{L}^{}|f(z)||dz|
    \end{align*}
    而:
    \begin{align*}
        |f(z)|=|x^2+iy^2|=\sqrt{x^4+y^4}\xrightarrow{x=0}\sqrt{y^4}=|y^2|\le 1
    \end{align*}
    并且有:
    \begin{align*}
        |dz|=ds
    \end{align*}
    所以有:
    \begin{align*}
        \left|\int_{L}^{}f(z)dz\right|\le \int_{L}^{}|f(z)||dz|\\
        \le 1\cdot \int_{-1}^{1}ds=1\cdot 2=2
    \end{align*}
\end{zm}
Task2:
\begin{zm}
    同Task1的思路有:
    \begin{align*}
        \left|\int_{L}^{}\frac{dz}{z^2}\right|\le \int_{L}^{}\left|\frac{1}{z^2}\right||dz|
    \end{align*}
    由于:
    \begin{align*}
        \left|\frac{1}{z^2}\right|=\frac{1}{|z|^2}\le 1
    \end{align*}
    并且有:
    \begin{align*}
        |dz|=ds
    \end{align*}
    所以有:
    \begin{align*}
        \left|\int_{L}^{}\frac{dz}{z^2}\right|\le \int_{L}^{}\left|\frac{1}{z^2}\right||dz|\\
        \le 1\cdot \int_{L}ds=1\cdot 1=1\le 2
    \end{align*}
\end{zm}
Task3:
\begin{zm}
    由于原点是奇点所以在此处无法使用柯西积分定理,考虑:
    \begin{align*}
        2\pi f(0)=\int_{0}^{2\pi}f(0)d\theta
    \end{align*}
    则即证明:
    \begin{align*}
        \lim_{r\to 0^+}\int_{0}^{2\pi}f(re^{i\theta})-f(0)d\theta=0
    \end{align*}
    由于函数在原点处解析所以,点$z=re^{i\theta}$即原点外半径为$r$的圆上的点,可以得到存在原点处的导数$f'(0)$使得:
    \begin{align*}
        \lim_{r\to 0^+}\frac{f(re^{i\theta})-f(0)}{re^{i\theta}}=f'(0)
    \end{align*}
    由于:
    \begin{align*}
        \lim_{r\to 0^+}\int_{0}^{2\pi}re^{i\theta}f'(0)d\theta=\lim_{r\to 0^+}r\cdot f'(0)\cdot \int_{0}^{2\pi}e^{i\theta}d\theta=0
    \end{align*}
    所以$\lim_{r\to 0^+}\int_{0}^{2\pi}f(re^{i\theta})-f(0)d\theta=0$成立。
\end{zm}
Task4:
\begin{tm}
    利用$Cauchy$积分定理证明复积分:\\
(1)对于积分$\int_{|z|=1}^{}\frac{dz}{z^2+2}$,由于函数$f(z)=\frac{1}{z^2+2}$的奇点为$z=\pm \sqrt{2}i$处于圆周$|z|=1$之外,所以由柯西积分定理:$\int_{|z|=1}^{}\frac{dz}{z^2+2}=0$\\
(2)对于积分$\int_{|z|=2}^{}\frac{dz}{z^2+2}$,由于函数$f(z)=\frac{1}{z^2+2}$的奇点在圆周$C:|z|=2$之内,考虑两个小圆周$C_1:|z-\sqrt{2}i|=r$以及$C_2:|z+\sqrt{2}i|=r$,方向均取逆时针为正方向,则函数在$C\cup C_1^-\cup C_2^-$所围成的多连通区域上是解析的,由柯西积分定理原积分:
\begin{align*}
    &\int_{C}^{}\frac{dz}{z^2+2}=\int_{C\cup C_1^-\cup C_2^-\cup C_1^+\cup C_2^+}^{}\frac{dz}{z^2+2}\\
    &=\int_{C\cup C_1^-\cup C_2^-}^{}\frac{dz}{z^2+2}+\int_{C_1^+}^{}\frac{dz}{z^2+2}+\int_{C_2^+}^{}\frac{dz}{z^2+2}\\
    &=0+\int_{C_1^+}^{}\frac{dz}{z^2+2}+\int_{C_2^+}^{}\frac{dz}{z^2+2}
\end{align*}
由于:
\begin{align*}
    &\int_{|z-a|=r}^{}\frac{dz}{z^2-a^2}\\
    &=\frac{1}{2a}\left(\int_{|z-a|=r}^{}\frac{dz}{z-a}-\int_{|z-a|=r}^{}\frac{dz}{z+a}\right)\\
    &=\frac{1}{2a}\left(2\pi i-0\right)=\frac{\pi i}{a}
\end{align*}
所以原积分为:
\begin{align*}
    0+\int_{C_1^+}^{}\frac{dz}{z^2+2}+\int_{C_2^+}^{}\frac{dz}{z^2+2}\\
    =0+\frac{\pi i}{\sqrt{2}i}+\frac{\pi i}{-\sqrt{2}i}=0
\end{align*}
(3)对于积分$\int_{|z|=1}^{}\frac{dz}{(2z+1)(z-2)}$,由于圆周$C:|z|=1$之内包含函数$g(z)=\frac{1}{(2z-1)(z-2)}$的奇点$z=-\frac{1}{2}$,所以考虑一个小圆周$C':|z+\frac{1}{2}|=r$,方向为逆时针,则同(2),原积分可以变为:
\begin{align*}
    \int_{|z|=1}^{}\frac{1}{5}\left(\frac{1}{z-2}-\frac{2}{2z+1}\right)dz\\
    =-\frac{1}{5}\int_{C'^+}^{}\left(\frac{2}{2z+1}\right)dz=-\frac{2\pi i}{5}
\end{align*}
\end{tm}
Task5:
\begin{zm}
    要求证明柯西积分公式:\\
    定义函数$F(\xi)=\frac{f(\xi)}{\xi-z}$,其中$z\in D$为一定点,作为$\xi$的函数,$F$在$D$上除了$\xi=z$点之外都解析。以$z$为中心,充分小的$\rho>0$为半径做一个顺时针方向的圆周$\gamma_\rho:|\xi-z|=\rho$,使得
    $\gamma_\rho$在$C_0^+$之内,在其余的$C_k^-$之外。令$\Gamma=C\cup \gamma_\rho=C_0+\cup C_1^-\cup C_2^-\cup \dots \cup C_n^-\cup \gamma_\rho^-$,记$\Gamma$围成的区域为$D'$,则$F$在$D'$上解析,且在$\Gamma\cup D'$上连续,对
    $F$在$D'$上应用柯西积分定理有:
    \begin{align*}
        &0=\int_{\Gamma}^{}F(\xi)d\xi=\int_{\Gamma}^{}\frac{f(\xi)}{\xi-z}d\xi\\
        &=\left(\int_{C_0^+}+\int_{C_1^-}+\dots +\int_{C_n^-}+\int_{\gamma_\rho^-}\right)F(\xi)d\xi\\
        &=\left(\int_{C}-\int_{\gamma_\rho^+}\right)F(\xi)d\xi
    \end{align*}
    所以可以得到:
    \begin{align*}
        \int_{C}\frac{f(\xi)}{\xi-z}d\xi=\int_{\gamma_\rho^+}\frac{f(\xi)}{\xi-z}d\xi
    \end{align*}
    由于:
    \begin{align*}
        \int_{\gamma_\rho^+}\frac{1}{\xi-z}d\xi=\int_{|\xi-z|=\rho}\frac{1}{\xi-z}d\xi=2\pi i
    \end{align*}
    故:
    \begin{align*}
        f(z)=\frac{1}{2\pi i}\int_{\gamma_\rho^+}\frac{f(z)}{\xi-z}d\xi
    \end{align*}
    故我们所需要证明的第一个等式即:
    \begin{align*}
        \int_{\gamma_\rho^+}\frac{f(z)}{\xi-z}d\xi=\int_{\gamma_\rho^+}\frac{f(\xi)}{\xi-z}d\xi
    \end{align*}
    考虑做差求解。事实上:
    \begin{align*}
        \left|\int_{\gamma_\rho^+}\frac{f(\xi)-f(z)}{\xi-z}d\xi\right|\le \int_{\gamma_\rho^+}\frac{|f(\xi)-f(z)|}{|\xi-z|}|d\xi|
    \end{align*}
    由于$f$在$D\cup C$上连续,即在一个闭的区域上连续,因而一直连续,故:
    \begin{align*}
        &\qquad \forall \epsilon>0, \exists \delta=\delta(\epsilon)>0,\quad s.t.\\
        &|w-z|<\delta (w,z\in\overline{D})\Rightarrow \left|f(w)-f(z)\right|<\epsilon.
    \end{align*}
    所以当$\rho<\delta$时,有$|\xi-z|=\rho<\delta\Rightarrow |f(\xi)-f(z)|<\epsilon$,所以:
    \begin{align*}
        \int_{\gamma_\rho^+}\left|\frac{f(\xi)-f(z)}{\xi-z}\right|\cdot |d\xi|\le \int_{\gamma_\rho^+}\frac{\epsilon}{\rho}|d\xi|\\
        =\frac{\epsilon}{\rho}\int_{\gamma_\rho^+}|d\xi|=\frac{\epsilon}{\rho}\cdot 2\pi \rho=2\pi \epsilon
        \end{align*}
    所以原积分为:
    \begin{align*}
        \left|\int_{\gamma_\rho^+}\frac{f(z)}{\xi-z}d\xi-\int_{\gamma_\rho^+}\frac{f(\xi)}{\xi-z}d\xi\right|\le 2\pi \epsilon\to 0(\epsilon\to 0^+)
    \end{align*}
    故而:
    \begin{align}
        f(z)=\frac{1}{2\pi i}\int_{\gamma_\rho^+}\frac{f(\xi)}{\xi-z}d\xi=\frac{1}{2\pi i}\int_{C}\frac{f(\xi)}{\xi-z}d\xi,\quad z\in D
    \end{align}
    下一步证明:$f$在$D$上内的各阶导数满足:
    \begin{align}
        f^(n)(z)=\frac{n!}{2\pi i}\int_{C}\frac{f(\xi)}{(\xi-z)^{n+1}}d\xi,\quad z\in D,n\in\mathbb{N}
    \end{align}
    当$n=1$时,即证明:
    \begin{align*}
        f'(z)=\frac{1}{2\pi i}\int_{C}^{}\frac{f(\xi)}{(\xi-z)^2}d\xi,z\in D
    \end{align*}
    我们有:
    \begin{align*}
        f'(z)=\lim_{\Delta z\to 0}\frac{f(z+\Delta z)-f(z)}{\Delta z}
    \end{align*}
    当$\Delta z$足够小时,我们有:
    \begin{itemize}
        \item $|\Delta z|>0$
        \item $(z+\Delta z)\in D$
    \end{itemize}
    有上一步的结论可知:
    \begin{align*}
        &\begin{cases}
            f(z)=\frac{1}{2\pi i}\int_{C}\frac{f(\xi)}{\xi-z}d\xi,\\
            f(z+\Delta z)=\frac{1}{2\pi i}\int_{C}\frac{f(\xi)}{\xi-(z+\Delta z)}d\xi
        \end{cases}\\
        &\Rightarrow \left|\frac{f(z+\Delta z)-f(z)}{\Delta z}\right|=\left|\frac{1}{2\pi i}\int_{C}\frac{f(\xi)}{(\xi-z)(\xi-z-\Delta z)}\right|\\
        &\le \frac{1}{2\pi}\left|\int_{C}\frac{|f(\xi)|\cdot |d\xi|}{|\xi-z|\cdot |\xi-z-\Delta z|}\right|
    \end{align*}
    由于$f$在$C$上连续,所以:
    \begin{align*}
        \exists M>0\space s.t.\space |f(\xi)|\le M,\forall \xi\in C.
    \end{align*}
    另外,$z$在开集$D$之内,故$z$到$D$的边界上的点$\xi$的最小距离$d>0$即:
    \begin{align*}
        0<d=\underset{\xi\in C}{inf}|\xi-z|=\underset{\xi\in C}{min}|\xi-z|
    \end{align*}
    所有有:
    \begin{align*}
       \begin{cases}
        |\xi-z|\ge d,\forall \xi\in C;\\
        |\xi-z-\Delta z|\ge |\xi-z|-|\Delta z|\\
        \ge d-|\Delta z|\overset{|\Delta z|<\frac{d}{2}}{\ge} d-\frac{d}{2}=\frac{d}{2}
       \end{cases}
    \end{align*}
    所有当$|\Delta z|<\frac{d}{2}$时,有:
    \begin{align*}
        &\left|\frac{f(z+\Delta z)-f(z)}{\Delta z}-\frac{1}{2\pi i}\int_{C}\frac{f(\xi)}{(\xi-z)^2}d\xi\right|\\
        &=\left|\frac{1}{2\pi i}\int_{C}\frac{f(\xi)}{(\xi-z)(\xi-z-\Delta z)}-\frac{1}{2\pi i}\int_{C}\frac{f(\xi)}{(\xi-z)^2}d\xi\right|\\
        &=\frac{1}{2\pi}\int_{C}\frac{|\Delta z|\cdot |f(\xi)|}{|(\xi-z)^2(\xi-z-\Delta z)|}d\xi\le \frac{M}{2\pi}\int_{C}\frac{|\Delta z|}{d^2\times \frac{d}{2}}|d\xi|\\
        &=\frac{M}{\pi d^3}|\Delta z|\cdot L
    \end{align*}
    其中$L$为$C$的弧长。所有当$\Delta z\to 0$时$|f'(z)-\frac{1}{2\pi i}\int_{C}\frac{f(\xi)}{(\xi-z)^2}d\xi|\le 0$,所以$n=1$时,结论成立,有了$f'(z)$之后,重复上面的讨论即得到对任意的$n\ge 1$结论成立
\end{zm}
Task6:
\begin{tm}
$C$为逆时针方向的圆周$|z|=2$求积分:
(1):对于积分$\int_{C}\frac{dz}{\cos(\frac{z}{2})}=\int_{C}\frac{\frac{z}{\cos \frac{z}{2}}dz}{z-0}$,因为函数$f(z)=\frac{z}{\cos \frac{z}{2}}$在圆盘$|z|<2$内部解析,且在$|z|\le 2$上连续,所以由柯西积分公式:
\begin{align*}
    \frac{1}{2\pi i}\int_{C}\frac{f(z)}{(z-0)}dz=f(0)=\frac{0}{\cos 0}=0
\end{align*}
所以原积分值为$0$
(2):对于积分$\int_{C}\frac{e^{\sin z}}{100z^2=1}dz=\int_{C}\frac{e^{\sin z}}{(10z+i)(10z-i)}=\frac{1}{20i}\int_{C}(\frac{e^{\sin z}}{z-\frac{i}{10}}-\frac{e^{\sin z}}{z+\frac{i}{10}})$,由于$f(z)=e^{\sin z}$在$C$的内部$|z|<2$解析,在$|z|\le 2$上连续,所以由柯西积分公式,原积分变为:
\begin{align*}
    \frac{1}{20i}\cdot 2\pi i\cdot (f(\frac{i}{10})-f(-\frac{i}{10}))=\frac{\pi}{5}\cdot \left(\frac{e^{\sin \frac{i}{10}}-e^{-\sin \frac{i}{10}}}{2}\right)
\end{align*}
因为:
\begin{align*}
    \sin\frac{i}{10}=\frac{e^{-\frac{1}{10}}-e^{\frac{1}{10}}}{2i}\\
    \sin\frac{-i}{10}=\frac{e^{\frac{1}{10}}-e^{-\frac{1}{10}}}{2i}\\
    \Rightarrow e^{\sin\frac{i}{10}}=e^{\frac{e^{\frac{1}{10}}-e^{-\frac{1}{10}}}{2}i}\\
    \overset{a=\frac{e^{\frac{1}{10}}-e^{-\frac{1}{10}}}{2}}{\Rightarrow}\cos a+i\sin a
\end{align*}
同理$e^{\sin\frac{i}{10}}=\cos a-i\sin a$
所以原积分为:
\begin{align*}
    \frac{\pi}{5}i\sin a=\frac{\pi}{5}i\sin\left(\sinh \left(\frac{1}{10}\right)\right)
\end{align*}
\end{tm}
Task7:
\begin{tm}
利用柯西积分公式求解复积分:
\begin{align*}
    \int_{C_j}\frac{\sin\left(\frac{\pi}{4}z\right)}{z^2-1}dz,\quad j=1,2,3
\end{align*}
(j=1):$C_1:|z+1|=\frac{1}{2}$,函数$f(z)=\frac{\sin\left(\frac{\pi}{4}z\right)}{z^2-1}$有奇点$z=\pm 1$。\\
原积分变为:
\begin{align*}
    \frac{1}{2}\cdot \int_{C_1}\left(\frac{\sin\left(\frac{\pi}{4}z\right)}{z-1}-\frac{\sin\left(\frac{\pi}{4}z\right)}{z+1}\right)dz
\end{align*}
由柯西积分定理,前一项为0,所以原积分为:
\begin{align*}
    -\frac{1}{2}\cdot \int_{C_1}\frac{\sin\left(\frac{\pi}{4}z\right)}{z+1}dz
\end{align*}
由于函数$g(z)=\sin\frac{\pi}{4}z$在$C_1$内部$D$解析,且在$C_1\cup D$上连续,所以由柯西积分公式,原积分变为:
\begin{align*}
    -\frac{1}{2}\cdot 2\pi i\cdot \sin\frac{-1\cdot \pi}{4}=\frac{\sqrt{2}}{2}\pi i
\end{align*}
(j=2):$C_2:|z-1|=\frac{1}{2}$,类似(j=1)分解可以得到原积分为:
\begin{align*}
    \frac{1}{2}\cdot \int_{C_2}\frac{\sin\left(\frac{\pi}{4}z\right)}{z-1}dz=\frac{1}{2}\cdot 2\pi i\cdot \sin\frac{\pi}{4}=\frac{\sqrt{2}}{2}\pi i
\end{align*}
(j=3):$C_3:|z|=4$则此时两个奇点都在$C_3$的内部。考虑$C_1,C_2$,则原积分变为:
\begin{align*}
    &\frac{1}{2}\cdot \int_{C_3\cup C_1^-\cup C_2^-\cup C_1^+\cup C_2^+}\left(\frac{\sin\left(\frac{\pi}{4}z\right)}{z-1}-\frac{\sin\left(\frac{\pi}{4}z\right)}{z+1}\right)dz\\
    &=\frac{1}{2}\cdot \int_{C_3\cup C_1^-\cup C_2^-}\left(\frac{\sin\left(\frac{\pi}{4}z\right)}{z-1}-\frac{\sin\left(\frac{\pi}{4}z\right)}{z+1}\right)dz\\
    &+\frac{1}{2}\cdot \int_{C_1^+\cup C_2^+}\left(\frac{\sin\left(\frac{\pi}{4}z\right)}{z-1}-\frac{\sin\left(\frac{\pi}{4}z\right)}{z+1}\right)dz
\end{align*}
由于函数$\left(\frac{\sin\left(\frac{\pi}{4}z\right)}{z-1}-\frac{\sin\left(\frac{\pi}{4}z\right)}{z+1}\right)$在$C_3\cup C_1^-\cup C_2^-$所包含的区域内解析,有柯西积分定理原积分变为:
\begin{align*}
    0+\frac{1}{2}\cdot\int_{C_1^+\cup C_2^+}\left(\frac{\sin\left(\frac{\pi}{4}z\right)}{z-1}-\frac{\sin\left(\frac{\pi}{4}z\right)}{z+1}\right)dz=\sqrt{2}\pi i
\end{align*}
\end{tm}
Task8:
\begin{zm}
    函数$f(z)=e^z$在$C:|z|=1$内部$D$解析,且在$D\cup C$上连续,所以由柯西积分公式:
    \begin{align*}
        \int_{|z|=1}\frac{e^z}{z}dz=\int_{C}\frac{e^z}{z-0}dz=2\pi i\cdot f(0)=2\pi i
    \end{align*}
    令$z=e^{i\theta},dz=ie^{i\theta}d\theta$则原积分变为:
    \begin{align*}
        &\int_{0}^{2\pi}\frac{e^{e^{i\theta}}}{e^{i\theta}}\cdot ie^{i\theta}d\theta=2\pi i\\
        &\Rightarrow \int_{0}^{2\pi}e^{\cos \theta+i\sin\theta}=2\pi\\
        &\Rightarrow \int_{0}^{2\pi}e^{\cos \theta}e^{i\sin\theta}d\theta=2\pi\\
        &\Rightarrow \int_{0}^{2\pi}e^{\cos\theta}\left(\cos\sin\theta+i\sin\sin\theta\right)d\theta=2\pi
    \end{align*}
    通过实部和虚部相对应可以得到:
    \begin{align*}
        &\int_{0}^{2\pi}e^{\cos \theta}\cos(\sin\theta)d\theta=2\pi\\
        &\int_{0}^{2\pi}e^{\cos \theta}\sin(\sin\theta)d\theta=0
    \end{align*}
    又令$u=2\pi-\theta$,有:
    \begin{align*}
        &2\pi=\int_{0}^{2\pi}e^{\cos \theta}\cos(\sin\theta)d\theta\\
        &=\int_{0}^{\pi}e^{\cos \theta}\cos(\sin\theta)d\theta+\int_{\pi}^{2\pi}e^{\cos \theta}\cos(\sin\theta)d\theta\\
        &=\int_{0}^{\pi}e^{\cos \theta}\cos(\sin\theta)d\theta+\int_{\pi}^{0}e^{\cos (2\pi-u)}\cos(\sin(2\pi-u))d(2\pi-u)\\
        &=2\int_{0}^{\pi}e^{\cos \theta}\cos(\sin\theta)d\theta\\
        &\Rightarrow \int_{0}^{\pi}e^{\cos \theta}\cos(\sin\theta)d\theta=\pi
    \end{align*}
\end{zm}
\end{document}

\documentclass{ctexart}
\usepackage{paracol}
\usepackage{newtxtext, geometry, amsmath, amssymb,yhmath}
\usepackage{graphicx}
\usepackage{amsthm} % 设置定理环境,要在amsmath后用
\usepackage{xpatch} % 用于去除amsthm定义的编号后面的点
\usepackage[strict]{changepage} % 提供一个 adjustwidth 环境
\usepackage{multicol} % 用于分栏
\usepackage{tikz}
\usetikzlibrary{arrows.meta,decorations.markings,patterns,calc}
\usepackage[fontsize=14pt]{fontsize} % 字号设置
\usepackage{esvect} % 实现向量箭头输入(显示效果好于\vec{}和\overrightarrow{}),格式为\vv{⟨向量符号⟩}或\vv*{⟨向量符号⟩}{⟨下标⟩} 
\usepackage[
            colorlinks, % 超链接以颜色来标识,而并非使用默认的方框来标识
            linkcolor=mlv,
            anchorcolor=mlv,
            citecolor=mlv % 设置各种超链接的颜色
            ]
            {hyperref} % 实现引用超链接功能
\usepackage{tcolorbox} % 盒子效果
\tcbuselibrary{most} % tcolorbox宏包的设置,详见宏包说明文档
\tcbuselibrary{breakable=forced}
\geometry{a4paper,centering,scale=0.8}

% ————————————————————————————————————自定义符号————————————————————————————————————
\newcommand{\。}{.} % 示例:将指令\。定义为全角句点。
\newcommand{\zte}{\mathrm{e}} % 正体e
\newcommand{\ztd}{\mathrm{d}} % 正体d
\newcommand{\zti}{\mathrm{i}} % 正体i
\newcommand{\ztj}{\mathrm{j}} % 正体j
\newcommand{\ztk}{\mathrm{k}} % 正体k
\newcommand{\CC}{\mathbb{C}} % 黑板粗体C
\newcommand{\QQ}{\mathbb{Q}} % 黑板粗体Q
\newcommand{\RR}{\mathbb{R}} % 黑板粗体R
\newcommand{\ZZ}{\mathbb{Z}} % 黑板粗体Z
\newcommand{\NN}{\mathbb{N}} % 黑板粗体N
% ————————————————————————————————————自定义符号————————————————————————————————————

% ————————————————————————————————————自定义颜色————————————————————————————————————
\definecolor{mlv}{RGB}{40, 137, 124} % 墨绿
\definecolor{qlv}{RGB}{240, 251, 248} % 浅绿
\definecolor{slv}{RGB}{33, 96, 92} % 深绿
\definecolor{qlan}{RGB}{239, 249, 251} % 浅蓝
\definecolor{slan}{RGB}{3, 92, 127} % 深蓝
\definecolor{hlan}{RGB}{82, 137, 168} % 灰蓝
\definecolor{qhuang}{RGB}{246, 250, 235} % 浅黄
\definecolor{shuang}{RGB}{77, 82, 59} % 深黄
\definecolor{qzi}{RGB}{240, 243, 252} % 浅紫
\definecolor{szi}{RGB}{49, 55, 70} % 深紫
% 将RGB换为rgb,颜色数值取值范围改为0到1
% ————————————————————————————————————自定义颜色————————————————————————————————————

% ————————————————————————————————————盒子设置————————————————————————————————————
% tolorbox提供了tcolorbox环境,其格式如下:
% 第一种格式:\begin{tcolorbox}[colback=⟨背景色⟩, colframe=⟨框线色⟩, arc=⟨转角弧度半径⟩, boxrule=⟨框线粗⟩]   \end{tcolorbox}
% 其中设置arc=0mm可得到直角;boxrule可换为toprule/bottomrule/leftrule/rightrule可分别设置对应边宽度,但是设置为0mm时仍有细边,若要绘制单边框线推荐使用第二种格式
% 方括号内加上title=⟨标题⟩, titlerule=⟨标题背景线粗⟩, colbacktitle=⟨标题背景线色⟩可为盒子加上标题及其背景线
\newenvironment{kuang1}{
    \begin{tcolorbox}[breakable,colback=hlan, colframe=hlan, arc = 1mm]
    }
    {\end{tcolorbox}}
\newenvironment{kuang2}{
    \begin{tcolorbox}[breakable,colback=hlan!5!white, colframe=hlan, arc = 1mm]
    }
    {\end{tcolorbox}}
% 第二种格式:\begin{tcolorbox}[enhanced, colback=⟨背景色⟩, boxrule=0pt, frame hidden, borderline={⟨框线粗⟩}{⟨偏移量⟩}{⟨框线色⟩}]   {\end{tcolorbox}}
% 将borderline换为borderline east/borderline west/borderline north/borderline south可分别为四边添加框线,同一边可以添加多条
% 偏移量为正值时,框线向盒子内部移动相应距离,负值反之
\newenvironment{kuang3}{
    \begin{tcolorbox}[breakable,enhanced, colback=hlan!5!white, boxrule=0pt, frame hidden,
        borderline south={0.5mm}{0.1mm}{hlan}]
    }
    {\end{tcolorbox}}
\newenvironment{lvse}{
    \begin{tcolorbox}[breakable,enhanced, colback=qlv, boxrule=0pt, frame hidden,
        borderline west={0.7mm}{0.1mm}{slv}]
    }
    {\end{tcolorbox}}
\newenvironment{lanse}{
    \begin{tcolorbox}[breakable,enhanced, colback=qlan, boxrule=0pt, frame hidden,
        borderline west={0.7mm}{0.1mm}{slan}]
    }
    {\end{tcolorbox}}
\newenvironment{huangse}{
    \begin{tcolorbox}[breakable,enhanced, colback=qhuang, boxrule=0pt, frame hidden,
        borderline west={0.7mm}{0.1mm}{shuang}]
    }
    {\end{tcolorbox}}
\newenvironment{zise}{
    \begin{tcolorbox}[breakable,enhanced, colback=qzi, boxrule=0pt, frame hidden,
        borderline west={0.7mm}{0.1mm}{szi}]
    }
    {\end{tcolorbox}}
% tcolorbox宏包还提供了\tcbox指令,用于生成行内盒子,可制作高光效果
\newcommand{\hl}[1]{
    \tcbox[on line, arc=0pt, colback=hlan!5!white, colframe=hlan!5!white, boxsep=1pt, left=1pt, right=1pt, top=1.5pt, bottom=1.5pt, boxrule=0pt]
{\bfseries \color{hlan}#1}}
% 其中on line将盒子放置在本行(缺失会跳到下一行),boxsep用于控制文本内容和边框的距离,left、right、top、bottom则分别在boxsep的参数的基础上分别控制四边距离
% ————————————————————————————————————盒子设置————————————————————————————————————

% ————————————————————————————————————自定义字体设置————————————————————————————————————
\setCJKfamilyfont{xbsong}{方正小标宋简体}
\newcommand{\xbsong}{\CJKfamily{xbsong}}
% CTeX宏集还预定义了\songti、\heiti、\fangsong、\kaishu、\lishu、\youyuan、\yahei、\pingfang等字体命令
% 由于未知原因,一些设备可能无法调用这些字体,故此文档暂时未使用
% ————————————————————————————————————自定义字体设置————————————————————————————————————

% ————————————————————————————————————各级标题设置————————————————————————————————————
\ctexset{
    % 修改 section。
    section={   
    % 设置标题编号前后的词语,使用name={⟨前部分⟩,⟨后部分⟩}参数进行设置。    
        name={\textbf{第},\textbf{章}\hspace{18pt}},
    % 使用number参数设置标题编号,\arabic设置为阿拉伯数字,\chinese设置为中文,\roman设置为小写罗马字母,\Roman设置为大写罗马字母,\alph设置为小写英文,\Alph设置为大写英文。
        number={\textbf{\chinese{section}}},
    % 参数format设置标题整体的样式。包括标题主题、编号以及编号前后的词语。
    % 参数format还可以设置标题的对齐方式。居中对齐\centering,左对齐\raggedright,右对齐\hfill
        format=\color{hlan}\centering\zihao{2}, % 设置 section 标题为正蓝色、黑体、居中对齐、小二号字
    % 取消编号后的空白。编号后有一段空白,参数aftername=\hspace{0pt}可以用来控制编号与标题之间的距离。
        aftername={}
    },
    % 修改 subsection。
    subsection={   
        name={\S \hspace{6pt}, \hspace{8pt}},
        number={\arabic{section}.\arabic{subsection}},
        format=\color{hlan}\centering\zihao{-2}, % 设置 subsection 标题为黑体、三号字
        aftername=\hspace{0pt}
    },
    subsubsection={   
        name={,、},
        number={\chinese{subsubsection}},
        format=\raggedright\color{hlan}\zihao{3},
        aftername=\hspace{0pt}
    },
    part={   
        name={第,部分},
        number={\chinese{part}},
        format=\color{white}\centering\bfseries\zihao{-1},
        aftername=\hspace{1em}
    }
}
% 各个标题设置的各行结尾一定要记得写逗号!!!!!!!!!!!!!!!!!!!!!!!!!!!!!!!!!!!!!!!!!!
% ————————————————————————————————————各级标题设置————————————————————————————————————

% ————————————————————————————————————定理类环境设置————————————————————————————————————
\newtheoremstyle{t}{0pt}{}{}{\parindent}{\bfseries}{}{1em}{} % 定义新定理风格。格式如下:
%\newtheoremstyle{⟨风格名⟩}
%                {⟨上方间距⟩} % 若留空,则使用默认值
%                {⟨下方间距⟩} % 若留空,则使用默认值
%                {⟨主体字体⟩} % 如 \itshape
%                {⟨缩进长度⟩} % 若留空,则无缩进;可以使用 \parindent 进行正常段落缩进
%                {⟨定理头字体⟩} % 如 \bfseries
%                {⟨定理头后的标点符号⟩} % 如点号、冒号
%                {⟨定理头后的间距⟩} % 不可留空,若设置为 { },则表示正常词间间距;若设置为 {\newline},则环境内容开启新行
%                {⟨定理头格式指定⟩} % 一般留空
% 定理风格决定着由 \newtheorem 定义的环境的具体格式,有三种定理风格是预定义的,它们分别是:
% plain: 环境内容使用意大利斜体,环境上下方添加额外间距
% definition: 环境内容使用罗马正体,环境上下方添加额外间距
% remark: 环境内容使用罗马正体,环境上下方不添加额外间距
\theoremstyle{t} % 设置定理风格
\newtheorem{dyhj}{\color{slv} 定义}[subsection] % 定义定义环境,格式为\newtheorem{⟨环境名⟩}{⟨定理头文本⟩}[⟨上级计数器⟩]或\newtheorem{⟨环境名⟩}[⟨共享计数器⟩]{⟨定理头文本⟩},其变体\newtheorem*不带编号
\newtheorem{dlhj}{\color{shuang} 定理}[subsection]
\newtheorem{lthj}{\color{szi} }
\newtheorem*{tmhj}{\color{slan} 解}
\newtheorem*{jiehj}{\color{slan} 解}
\newtheorem*{zmhj}{\color{slan} 证明}
\newtheorem*{smhj}{\color{slan} }
\newenvironment{dy}{\begin{lvse}\begin{dyhj}}{\end{dyhj}\end{lvse}}
\newenvironment{zm}{\begin{lanse}\begin{zmhj}}{$\hfill \square$\end{zmhj}\end{lanse}}
\newenvironment{dl}{\begin{huangse}\begin{dlhj}}{\end{dlhj}\end{huangse}}
\newenvironment{lt}{\begin{zise}\begin{lthj}}{\end{lthj}\end{zise}}
\newenvironment{tm}{\begin{lanse}\begin{tmhj}}{\end{tmhj}\end{lanse}}
\newenvironment{sm}{\begin{lanse}\begin{smhj}}{\end{smhj}\end{lanse}}
% ————————————————————————————————————定理类环境设置————————————————————————————————————

% ————————————————————————————————————目录设置————————————————————————————————————
\setcounter{tocdepth}{4} % 设置在 ToC 的显示的章节深度
\setcounter{secnumdepth}{3} % 设置章节的编号深度
% 数值可选选项:-1 part 0 chapter 1 section 2 subsection 3 subsubsection 4 paragraph 5 subparagraph
\renewcommand{\contentsname}{\bfseries 目\hspace{1em}录}
% ————————————————————————————————————目录设置————————————————————————————————————

\everymath{\displaystyle} % 设置所有数学公式显示为行间公式的样式

\begin{document}

\quad\\
{
\Huge \bfseries 
\begin{kuang3}
    \color{hlan}
    \centering
    复变函数论第十次作业\\
    20234544 毛华豪
\end{kuang3}
}

\thispagestyle{empty} % 取消这一页的页码

% ————————————————————————————————————页码设置————————————————————————————————————
% \pagenumbering{Roman}
% \setcounter{page}{1}

% \begin{multicols}{2} % 设置环境中内容为两栏。数字为栏数
%     {
%     \tableofcontents
%     }
% \end{multicols}

% \newpage
% \setcounter{page}{1}
% \pagenumbering{arabic}
% ————————————————————————————————————页码设置————————————————————————————————————

% 指令\addcontentsline{toc}{⟨章节层级⟩}{⟨标题名⟩}可将“⟨标题名⟩”加入目录。如为无章节层级的标题,可先用\phantomsection指令添加section分级

% \begin{kuang1}
%     \part{第一部分的标题}
% \end{kuang1}

% \begin{kuang2}
%     \section{第一章的标题}
% \end{kuang2}

% \begin{kuang3}
%     \subsection{第一节的标题}
% \end{kuang3}

%\subsubsection{第一个大标题}


% \begin{dy} % 开始定义环境,格式为\begin{⟨环境名⟩}[⟨定理名⟩]
% 测试
% $$\int_{a}^{b} f(x)dx = F(b)-F(a) $$
% \end{dy}

% \begin{dl}
% 测试
% $$\int_{a}^{b} f(x)dx = F(b)-F(a) $$
% \end{dl}

% \begin{lt}
% 测试
% $$\int_{a}^{b} f(x)dx = F(b)-F(a) $$
% \end{lt}
Task1:
\begin{zm}
    第一步证明解析函数的泰勒展开:\\
    假设函数 $f$ 在区域 $D\subset \mathbb{C}$ 上解析,$a\in D$,且圆盘 $K=\{z\in\mathbb{C}:|z-a|<R\}\subset D$,则 $f$ 可展开为泰勒级数:
    \begin{align*}
        f(z)=\sum_{n=0}^{\infty}a_n(z-a)^n
    \end{align*}
    其中泰勒系数满足:
    \begin{align*}
        a_n=\frac{1}{2\pi i}\int\limits_{|z-a|=r\in(0,R)}\frac{f(z)}{(z-a)^{n+1}}dz=\frac{f^{(n)}(a)}{n!}
    \end{align*}
    积分路径方向为逆时针。此外,该展开是唯一的。\\
    任取一点 $z\in K$,则存在 $r\in (0,R)$ 使得 $z\in \{w:|w-a|<r\}$。应用柯西积分公式有:
    \begin{align*}
        f(z)=\frac{1}{2\pi i}\int_{|w-a|=r}\frac{f(w)}{w-z}dw
    \end{align*}
    积分路径为逆时针方向。
    
    \begin{center}
        \begin{tikzpicture}[scale=1.5, >=Stealth]
            \draw[->] (-3,0) -- (3,0) node[right] {$\mathrm{Re}$};
            \draw[->] (0,-2) -- (0,2.5) node[above] {$\mathrm{Im}$};
            \draw[dashed, fill=gray!10] plot[smooth cycle] coordinates {(-2,1.5) (1,2) (2.5,0.5) (2,-1) (0,-1.5) (-2,-0.5)};
            \node at (-1.5,1.8) {$D$};
            \draw[thick] (0,0) circle (1.2);
            \filldraw (0,0) circle (0.8pt) node[below left] {$a$};
            \node at (-0.8,1.2) {$K$};
            \node at (0.6,0) {$R$};
            \draw[->] (0,0) -- (1.2,0);
            \draw[thick, blue] (0,0) circle (0.7);
            \draw[blue, ->] (0,0) -- (0,0.7);
            \filldraw (0,0) circle (0.8pt) node[below left] {$a$};
            \node at (-0.2,0.35) {$r$};
            \draw[blue, ->] (1.2,-0.8) -- (0.55,-0.35);
            \node[blue, align=left] at (2,-1) {$z \in \{w:|w-a|<r\}$};
            \filldraw[red] (0.3,0.4) circle (0.8pt) node[above right] {$z$};
            \draw[<->] (-2.5,-1.8) -- (-1.5,-1.8) node[midway,below] {比例尺};
        \end{tikzpicture}
    \end{center}
    
    于是得到如下等式:
    \begin{align*}
        \frac{f(w)}{w-z}&=\frac{f(w)}{(w-a)-(z-a)}\\
        &=\frac{f(w)}{w-a}\cdot\frac{1}{1-\frac{z-a}{w-a}}\\
        &\stackrel{\sum_{n=0}^{\infty}\lambda^n=\frac{1}{1-\lambda}}{\underset{|\lambda|<1}{=}}\frac{f(w)}{w-a}\sum_{n=0}^{\infty}\left(\frac{z-a}{w-a}\right)^n
    \end{align*}
    注意到 $\sum_{n=0}^{\infty}\left(\frac{z-a}{w-a}\right)^n$ 满足:
    \begin{align*}
        \left|\frac{z-a}{w-a}\right|=\frac{|z-a|}{r}\triangleq s<1
    \end{align*}
    由于 $\sum_{n=0}^{\infty}\left(\frac{|z-a|}{r}\right)^n$ 收敛,故 $\sum_{n=0}^{\infty}\left(\frac{z-a}{w-a}\right)^n$ 一致收敛,因此有:
    \begin{align*}
        f(z)&=\frac{1}{2\pi i}\int\limits_{|w-a|=r}\frac{f(w)}{w-z}dw=\frac{1}{2\pi i}\int\limits_{|w-a|=r}\frac{f(w)}{w-a}\sum_{n=0}^{\infty}\left(\frac{z-a}{w-a}\right)^n dw\\
        &\overset{UC}{=}\frac{1}{2\pi i}\sum_{n=0}^{\infty}(z-a)^n\int\limits_{|w-a|=r}\frac{f(w)}{(w-a)^{n+1}}dw\triangleq \sum_{n=0}^{\infty}a_n(z-a)^n
    \end{align*}
    系数 $a_n$ 满足:
    \begin{align*}
        a_n&=\frac{1}{2\pi i}\int\limits_{|w-a|=r}\frac{f(w)}{(w-a)^{n+1}}dw\\
        &\overset{Cauchy}{=}\frac{f^{(n)}(a)}{n!},\quad n=0,1,2\dots
    \end{align*}
    
    最后证明展开的唯一性。若 $f$ 在 $K$ 上有另一展开式:
    \begin{align*}
        f(z)=\sum_{n=0}^{\infty}b_n(z-a)^n,\quad z\in K
    \end{align*}
    则对 $n\geq 0$ 有:
    \begin{align*}
        f(a)=b_0,f'(a)=b_1,f''(a)=2!b_2,\dots,f^{(n)}(a)=n!b_n
    \end{align*}
    因此 $b_n=\frac{f^{(n)}(a)}{n!}=a_n$,这说明展开式是唯一的。
    下面再证明此结论:
        $f$ 在 $D$ 上解析 $\Leftrightarrow$ 对于任意 $a\in D$,$f$ 可在某邻域内展开为关于 $(z-a)$ 的泰勒级数。\\
    $\Rightarrow:$ 由前述解析函数的泰勒展开直接可得。\\
    $\Leftarrow:$ 设 $f$ 在圆盘 $K=\{z\in\mathbb{C}:|z-a|<R\}$ 上可展开为 $f(z)=\sum_{n=0}^{\infty}a_n(z-a)^n$。根据幂级数性质(在收敛域内内闭一致收敛),$\sum_{n=0}^{\infty}a_n(z-a)^n$ 在 $|z-a|<R$ 上内闭一致收敛于 $f(z)$。由于每一项 $a_n(z-a)^n$ 都是解析函数,根据$Weierstrass$定理,$f$ 在 $|z-a|<R$ 上解析,因此在 $D$ 上解析。
\end{zm}
Task2:
\begin{zm}
    证明解析函数的零点孤立性:\\
    设 $f$ 在 $\{z\in\mathbb{C}:|z-a|<r\}$ 上解析,且 $a$ 是 $f$ 的零点。若 $f$ 在 $|z-a|<r$ 上不恒为零,则存在 $a$ 的某个邻域内除 $a$ 外不含 $f$ 的其他零点,即 $a$ 是 $f$ 的孤立零点。\\
    根据泰勒定理,$f$ 可表示为
    \begin{align*}
        f(z)=\sum_{n=0}^{\infty}a_n(z-a)^n,\quad |z-a|<r
    \end{align*}
    由于 $f$ 在 $|z-a|<r$ 上不恒为零,可设 $a$ 是 $f$ 的 $m$ 阶零点,即
    \begin{align*}
        f(z)=(z-a)^m\phi(z)
    \end{align*}
    其中 $\phi$ 在 $|z-a|<r$ 上解析且 $\phi(a)\neq 0$。特别地,$\phi$ 在 $a$ 点连续,故 $|\phi|$ 在 $a$ 点连续且 $|\phi(a)|>0$,因此存在邻域 $O(a,\delta)$ 使得
    \begin{align*}
        |f(z)|=|z-a|^m\cdot |\phi(z)|>0,\quad z\in O(a,\delta)\backslash \{a\}
    \end{align*}
    所以$a$孤立零点。
\end{zm}
Task3:
\begin{zm}
    因为$f$在区域$D$上解析,$z_0\in D$。由泰勒定理,考虑函数在$a_0$点附近的$Taylor$展开。$f(z)=C_0+C_1(z-a_0)+C_2(z-a_0)^2+\dots+C_n(z-a_0)^n+\dots$.代入$f^{(n)}(z_0)=0,n=1,2,\dots$可以得到$C_1=C_2=\dots=C_n=\dots=0$所以函数$f=C_0$为常值函数。
\end{zm}
Task4:
\begin{zm}
    考虑解析函数零点孤立性的推论:
    设函数 $f$ 在圆盘 $|z-a|<r$ 上解析,且存在一列零点 $\{z_n\}_{n=1}^\infty$ 收敛于 $a$($z_n\neq a,n=1,2,\dots$),则 $f$ 在 $|z-a|<r$ 上恒为零。\\
    由于 $f$ 在 $a$ 点解析,故在 $a$ 点连续:
    \begin{align*}
        f(a)=\lim_{n\to\infty}f(z_n)=0,\quad\forall n\ge 1
    \end{align*}
    因此 $a$ 是 $f$ 的零点。但 $z_n\to a(n\to\infty)$ 表明 $a$ 不是孤立零点,根据零点孤立性定理可得 $f\equiv 0$ 在 $|z-a|<r$ 上成立。\\
    设 $f$ 和 $g$ 都在区域 $D$ 上解析,且存在点列 $\{z_n\}_{n=1}^{\infty}$($z_n\neq a$, $n=1,2,\dots$)使得 $f(z_n)=g(z_n)$($n=1,2,\dots$),则 $f(z)=g(z)$ 在 $D$ 上成立。

    \begin{center}
        \begin{tikzpicture}[scale=1.5]
            \draw[dashed, pattern=north east lines, pattern color=gray!20] (0,0) ellipse (2.7 and 1.7);
            \node at (-2.5,1.3) {$D$};
            \filldraw (1,0.5) circle (1pt) node[below right]{$b$};
            \filldraw (-1.5,-0.5) circle (1pt) node[above left]{$a$};
            \draw[red, solid, thick] (1,0.5) -- 
                node[pos=0.3, above]{$L$} (0.2,0.4) -- 
                (0.2,-0.4) -- (-0.7,-1) -- (-1.5,-0.5);
        \end{tikzpicture}
    \end{center}
    由于 $D$ 是连通开集,任取 $a,b\in D$ 并用折线 $L$ 连接。设 $L$ 到边界 $\partial D$ 的距离为
    \begin{align*}
        d=\inf|z-w|>0,\quad z\in L,w\in \partial D
    \end{align*}
    
    \begin{center}
        \begin{tikzpicture}[scale=1.5]
            \def\r{1.25}      % 半径
            \def\d{0.9}    % 圆心距离(d < r)
            \draw (0,0) circle (\r) node[below]{$a_0$};
            \draw (\d,0) circle (\r) node[below]{$a_1$};
            \filldraw (0,0) circle (1pt);
            \filldraw (\d,0) circle (1pt);
            \draw[<->, red, solid, thick] (0,0*\r) -- node[below]{\scriptsize $d < r$} (\d,0*\r);
            \draw[->] (0,0) -- node[above]{$r$} (60:\r);
            \draw[->] (\d,0) -- node[above]{$r$} ($(\d,0)+(120:\r)$);
        \end{tikzpicture}
    \end{center}
    
    在 $L$ 上取分点 $a=a_0,a_1,\dots,a_n=b$ 使得相邻两点距离小于 $r\in(0,d)$。根据前述定理,$f-g\equiv 0$ 在 $|z-a_0|<r$ 上成立。由于 $|a_1-a_0|<r$,前两个圆盘交集非空。重复此过程可得 $f(b)=g(b)$。因$b$位置任意选取,故 $f\equiv g$ 于 $D$。
    
    (详细证明:令 $h(z)=f(z)-g(z)$,由 $h(z_n)=0$ 且 $z_n\to a$,根据推论得 $h\equiv 0$ 于 $|z-a_0|<r$。在 $a_0$ 与 $a_1$ 连线上取点列 $\{w_n\}\subset \{z:|z-a_1|<r\}$ 使得 $h(w_n)=0$ 且 $w_n\to a_1$,再次应用推论得 $h\equiv 0$ 于 $|z-a_1|<r$。重复此过程最终得 $h(b)=0$,由 $b$ 的任意性知 $f\equiv g$ 于 $D$。)
\end{zm}
Task5:
\begin{zm}
    不存在这样的函数,因为$\frac{1}{2k}\to 0$,$k$取$1,2\dots$时为一列趋向于原点的点列,满足$f(\frac{1}{2k})=0$,在原点解析意味着在原点的某一个领域内解析,假设在开圆盘$|z|<\rho$上解析,则由零点孤立性的推论可以知道在$|z|<\rho$上恒为0,取$k>\frac{1}{2\rho}+\frac{1}{2},k\in\mathbb{N}$则有$\frac{1}{2k-1}<\rho$,所以此时$f(\frac{1}{2k-1})=0\neq 1$。所以不存在这样的解析函数。
\end{zm}
Task6:
\begin{tm}
    指出下面两个函数在零点$z=0$点的阶数。\\
    (1):$z^2(e^{z^2}-1)=z^2\left(\sum_{n=0}^{\infty}\frac{(z^2)^n}{n!}-1\right)=z^4\left(\sum_{n=0}^{\infty}\frac{z^{2(n-1)}}{n!}\right)$,所以在$z=0$处的阶数为4.\\
    (2):$6\sin(z^3)+z^3(z^6-6)=6(z^3-\frac{z^9}{3!}+\frac{z^{15}}{5!}+\dots)+z^9-6z^3=z^{15}\left(\sum_{n=0}^{\infty}\frac{6z^{3n}}{(2n+5)!}\right)$,所以阶数为15.
\end{tm}
Task7:
\begin{zm}
    原函数$f(z)$可以写成$a_0+\sum_{n=1}^{\infty}a_nz^n$,记$\sum_{n=1}^{\infty}a_nz^n=g(z)$,因为幂级数的收敛半径$R>r$,所以可知:
    \begin{align*}
        a_n=\frac{1}{2\pi i}\int_{|z|=r}\frac{f(z)}{z^{n+1}}dz\\
        \Rightarrow |a_n|\le \frac{1}{2\pi}\int_{|z|=r}\frac{M}{r^{n+1}}dz\\
        =\frac{1}{2\pi}\cdot \frac{M}{r^{n+1}}\cdot 2\pi r=\frac{M}{r^{n}}
    \end{align*}
    有了系数之后可得到:
    \begin{align*}
        &|g(z)|\le \sum_{n=1}^{\infty}|a_n||z|^n\le \sum_{n=1}^{\infty}\frac{M}{r^n}|z|^n\\
        &=M\sum_{n=1}^{\infty}\left(\frac{|z|}{r}\right)^n=M\cdot \frac{|z|/r}{1-|z|/r}\\
        &=\frac{M|z|}{r-|z|}
    \end{align*}
    当$|z|<\frac{|a_0|}{|a_0|+M}r$时有:
    \begin{align*}
        \frac{M|z|}{r-|z|}<\frac{\frac{M|a_0|r}{|a_0|+M}}{r-\frac{|a_0|}{|a_0+M|r}}=|a_0|\Rightarrow |g(z)|<|a_0|
    \end{align*}
    所以$f(z)=a_0+g(z)\neq 0$(否则$|a_0|=|g(z)|$)
\end{zm}
Task8:
\begin{zm}
    设在圆盘$\{z\in\mathbb{C}:|z|<R\}$,解析函数$f(z)$的$Taylor$展开式为$f(z)=a_0+a_1z+a_2z^2+\dots+a_nz^n+\dots$,当$0\le r<R$时,积分$\frac{1}{2\pi}\int_{0}^{2\pi}|f(re^{i\theta})|^2d\theta$可以写为:
    \begin{align*}
        &\frac{1}{2\pi}\int_{0}^{2\pi}|f(re^{i\theta})|^2d\theta\\
        &=\frac{1}{2\pi}\int_{0}^{2\pi}f(re^{i\theta})\cdot\overline{f(re^{i\theta})}d\theta.
    \end{align*}
    将$f(z)$解析式代入,原积分变为:
    \begin{align*}
        &\frac{1}{2\pi}\int_{0}^{2\pi}(a_0+a_1re^{i\theta}+a_2re^{2i\theta}+\dots)\cdot\overline{(a_0+a_1re^{i\theta}+a_2re^{2i\theta}+\dots)}d\theta\\
        =&\frac{1}{2\pi}\int_{0}^{2\pi}(\sum_{n=0}^{\infty}|a_n|r^{2n}+(A_1re^{i\theta})+(\bar{A_1}re^{-i\theta})+\dots+(A_nre^{ni\theta})+\dots)
    \end{align*}
    因为在收敛域内部,幂级数是内闭一致收敛的,可以逐项积分,又因为:
    \begin{align*}
        \frac{1}{2\pi}\int_{0}^{2\pi}r^ke^{ni\theta}d\theta=\frac{r^k}{2\pi}\int_{0}^{2\pi}e^{ni\theta}d\theta=0
    \end{align*}
    所以原积分为:
    \begin{align*}
        &\frac{1}{2\pi}\int_{0}^{2\pi}(\sum_{n=0}^{\infty}|a_n|^2 r^{2n}+(A_1re^{i\theta})+(\bar{A_1}re^{-i\theta})+\dots+(A_nre^{ni\theta})+\dots)\\
        =&\frac{1}{2\pi}\int_{0}^{2\pi}(\sum_{n=0}^{\infty}|a_n|^2 r^{2n})+0+\dots+0+\dots=\sum_{n=0}^{\infty}|a_n|^2 r^{2n}
    \end{align*}
\end{zm}
Task9:
\begin{zm}
    设$f$为一个整函数,即在整个复平面上解析的函数,且存在正整数$n$以及两个正数$R$和$M$使得:
    \begin{align*}
        |f(z)|\le M|z|^n,\quad |z|\ge R
    \end{align*}
    因为$f$为整函数,所以在原点展开成泰勒级数得到:
    \begin{align*}
        f(z)=a_0+a_1z+a_2z^2+\dots+a_nz^n+\dots
    \end{align*}
    其中$a_k$满足:
    \begin{align*}
        |a_k|\le \frac{max_{|z|=r}|f(z)|}{r^k},\quad r>R
    \end{align*}
    由题设可知,当$|z|=r>R$时,满足$|f(z)|\le Mr^n$,故$|a_k|\le \frac{Mr^n}{r^k}=Mr^{n-k}$,当$k>n$时,令$r\to\infty$有$r^{n-k}\to 0$从而$|a_k|=0$,所以$a_k=0,k>n$,所以这样的函数一定是一个次数最大为$n$的多项式或者常数函数。
\end{zm}
Task10:
\begin{zm}
    这样的函数存在,考虑函数$f(z)=\frac{z}{z+1}$,则函数满足:
    \begin{itemize}
        \item $f(\frac{1}{n})=\frac{\frac{1}{n}}{1+\frac{1}{n}}=\frac{1}{n+1},\quad n=1,2\dots$.
        \item 函数在原点有定义并且解析
    \end{itemize}
    (前面我们提到了函数在某个区域$D$上解析的充要条件是对于区域内的任意一个点$a$都可以在某个收敛域内展开成关于$(z-a)$的幂级数。对于一个点的解析来说,充要条件为在该点可以展开成幂级数。函数$f(z)$可以展开成$f(z)=\frac{z}{z+1}=z\cdot \frac{1}{z+1}=z\cdot\sum_{n=0}^{\infty}z^n=\sum_{n=0}^{\infty}(-1)^nz^{n+1},\quad |z|<1$,所以函数解析并且满足条件,所以存在这样的函数。)
\end{zm}
Task11:
\begin{tm}
    计算积分:$\int_{C}\frac{z^3dz}{(z-3)(z^2-1)}$,在圆周$|z|=2$内部有奇点$\pm 1$,所以积分可以变为:
    \begin{align*}
        &\frac{1}{2}\int_{C}\frac{z^3dz}{(z-1)(z-3)}-\frac{1}{2}\int_{C}\frac{z^3dz}{(z+1)(z-3)}\\
        =&\frac{1}{2}\int_{C}\frac{\frac{z^3}{(z-3)}dz}{(z-1)}-\frac{1}{2}\frac{\frac{z^3}{(z-3)}dz}{(z+1)}
    \end{align*}
    令$g(z)=\frac{z^3}{z-3}$由柯西积分公式有原积分为:
    \begin{align*}
        \frac{1}{2}\cdot 2\pi i\cdot g(1)-\frac{1}{2}\cdot 2\pi i\cdot g(-1)=-\frac{3}{4}\pi i.
    \end{align*}
\end{tm}
\end{document}
